\section{Klassifikation}

Da die Software im Laufe des Projektes stetig weiterentwickelt wurde, steht zur Berechnung der Kosten bzw. Wahrscheinlichkeiten nur die am Besten funktionierende Methode zur Verfügung, \en{HMM mit maximum Approximation}.

\note{Bei Einzelwörtern hat der Sprecher ein besseres Ergebnis als All. Bei Wortfolgen schneidet das Modell All besser ab, als der eigentliche Sprecher!}

\subsection{Einzelworterkennung}

Die folgenden Tabellen zeigen ...


\begin{table}[!hp]
\centering
%\rowcolors{1}{RoyalBlue!20}{RoyalBlue!5}
\begin{tabular}{|l|l|l|l|}
\hline
Stillemodell 			& Referenzmodell 	  & Getestetes Wort 	& Kosten \\ \hline \hline
\rowcolor{green!50} A409& Drei\_Andreas\_A409 & Drei\_Andreas 	& -168,04 \\ \cline{3-4}
						&				 	  & Drei\_Gustav  	&  198,98 \\ \cline{3-4}
						&					  & Drei\_Lisa	  	&  113,26 \\ \hline \hline
A409					& Drei\_All\_409	  & Drei\_Andreas 	& -95,74 \\ \cline{3-4}
						&				 	  & Drei\_Gustav  	& -46,10 \\ \cline{3-4}
\rowcolor{green!50}		&					  & Drei\_Lisa	  	& -112,95 \\ \hline \hline
\rowcolor{green!50} G117& Drei\_Andreas\_A409 & Drei\_Andreas 	& -179,47 \\ \cline{3-4}
						&				 	  & Drei\_Gustav  	& 89,10\\ \cline{3-4}
						&					  & Drei\_Lisa	  	& 12,85 \\ \hline \hline
G117					& Drei\_All\_A409	  & Drei\_Andreas 	& -105,82 \\ \cline{3-4}
						&				 	  & Drei\_Gustav  	& -57,20 \\ \cline{3-4}
\rowcolor{green!50}		&					  & Drei\_Lisa	  	& -125,85  \\ \hline \hline
\end{tabular}
\caption[Getestete Prädiktion für das Stillemodell A409]{Getestete Prädiktion für das Stillemodell A409}
\label{tab:predict_G117}
\end{table}


\begin{table}[!hp]
\centering
\rowcolors{1}{RoyalBlue!20}{RoyalBlue!5}
\begin{tabular}{|l|l|l|l|l|}
\hline
Stillemodell & Getestet	für 	 			 & Wort 	 & Sprecher & Kosten \\\hline\hline
A409        & Signalverarbeitung\_Gustav\_G117 & Signalverarbeitung & Gustav &  230,60		\\ \hline
G117		& Signalverarbeitung\_Gustav\_G117 & Signalverarbeitung	& Gustav & 225,27	\\ \hline
A409		& Zwei\_Lisa\_A409				 & Zwei & Lisa & -210,28	\\ \hline
G117		& Zwei\_Lisa\_A409				 & Zwei & Lisa & -235,00 \\ \hline
\end{tabular}
\caption[Prädiktion mit Sprechererkennung]{Prädiktion mit Sprechererkennung}
\label{tab:predict_all_modells}
\end{table}

\newpage
\begin{table}[!hp]
\centering
\rowcolors{1}{RoyalBlue!20}{RoyalBlue!5}
\begin{tabular}{|l|l|l|l|}
\hline
Stillemodell & Getestet	für 	 			 & Wort 	 & Kosten \\\hline\hline
A409        & Signalverarbeitung\_Gustav\_G117 & Signalverarbeitung  & 447,73  		\\ \hline
G117		& Signalverarbeitung\_Gustav\_G117 & Signalverarbeitung & 442,40	\\ \hline
A409		& Zwei\_Lisa\_A409			& Zwei & 28,13	\\ \hline
G117		& Zwei\_Lisa\_A409			& Zwei & 21,17	\\ \hline
\end{tabular}
\caption[Prädiktion mit veralgemeinerten Modellen]{Prädiktion mit veralgemeinerten Modellen}
\label{tab:predict_ALL}
\end{table}



\bigskip
\subsection{Wortfolgenerkennung}

Da es wie bei der Einzelworterkennung kaum einen Unterschied zwischen dem verwendetem Stillemodell gibt, werden für die folgenden Tests nur G117 verwendet.
Aufgrund der sonst übergroßen Breite der Tabellen wird das Wort \en{Fouriertransformation} mit FT und \en{Signalverarbeitung} mit SV abgekürzt.

\begin{table}[!hp]
\centering
\rowcolors{1}{RoyalBlue!20}{RoyalBlue!5}
\begin{tabular}{|l||l|l|l|}
\hline
Gesprochenes Wort & Andreas & Gustav  & Lisa 	\\\hline\hline
Lisa     		   & Lisa 	 & Lisa	   & Lisa	\\ \hline
und		  		   & und  	 & und	   & und	\\ \hline
Gustav			   & Gustav  & Gustav  & Gustav	\\ \hline
und				   & 	 	 & und	   & und	\\ \hline
Andreas			   & Andreas & Andreas & Andreas\\ \hline
mögen			   & mögen	 & mögen   & mögen	\\ \hline
SV				   & SV		 & SV	   & SV	\\ \hline
\end{tabular}
\caption[Prädiktion mit veralgemeinerten Modellen]{Prädiktion mit veralgemeinerten Modellen}
\label{tab:predict_ALL_1}
\end{table}

\begin{table}[!hp]
\centering
\rowcolors{1}{RoyalBlue!20}{RoyalBlue!5}
\begin{tabular}{|l||l|l|l|l|l|l|}
\hline
Gesprochenes Wort & \multicolumn{2}{|c|}{Andreas} & \multicolumn{2}{|c|}{Gustav} & \multicolumn{2}{|c|}{Lisa} \\ \hline
		& Wort	 & Sprecher& Wort	& Sprecher & Wort 	& Sprecher\\	\hline \hline
Lisa    & Lisa 	 & Andreas & Lisa	& Gustav   & Lisa	& Lisa	\\ \hline
und		& und  	 & Andreas & und	& Gustav   & und	& Lisa	\\ \hline
Gustav	& Gustav & Andreas & Gustav & Gustav   & Gustav	& Lisa	\\ \hline
und		& 	 	 & 		   & und	& Gustav   & und	& Lisa	\\ \hline
Andreas	& Andreas& Andreas & Andreas& Andreas  & Andreas& Andreas \\ \hline
mögen	& mögen	 & Gustav  & mögen  & Gustav   & mögen	& Lisa	\\ \hline
SV		& SV	 & Andreas & SV	    & Gustav   & SV		& Lisa	\\ \hline
\end{tabular}
\caption[Prädiktion mit Sprechererkennung]{Prädiktion mit Sprechererkennung}
\label{tab:predict_Speaker_1}
\end{table}

\newpage

\begin{table}[!hp]
\centering
\rowcolors{1}{RoyalBlue!20}{RoyalBlue!5}
\begin{tabular}{|l||l|l|l|}
\hline
Gesprochenes Wort & Andreas & Gustav & Lisa 		\\\hline\hline
				   & 		 & moegen & drei	\\ \hline
und       		   & und 	 & und	  & und		\\ \hline
und				   & und  	 & und	  & und			\\ \hline
				   & 		 & drei	  & \\ \hline
FT 				   & FT 	 & FT 	  & FT\\ \hline
zwei			   & zwei 	 & zwei   & zwei	\\ \hline
drei			   & drei	 & drei   & drei	\\ \hline
\end{tabular}
\caption[Prädiktion mit veralgemeinerten Modellen]{Prädiktion mit veralgemeinerten Modellen}
\label{tab:predict_ALL_2}
\end{table}

\begin{table}[!hp]
\centering
\rowcolors{1}{RoyalBlue!20}{RoyalBlue!5}
\begin{tabular}{|l||l|l|l|l|l|l|}
\hline
Gesprochenes Wort & \multicolumn{2}{|c|}{Andreas} & \multicolumn{2}{|c|}{Gustav} & \multicolumn{2}{|c|}{Lisa} \\\hline
	& Wort & Sprecher & Wort & Sprecher & Wort   & Sprecher \\ \hline \hline
	& und  & Andreas  & drei & Andreas  & zwei   & Andreas	\\ \hline
und & und  & Gustav	  & und	 & Gustav   & und    & Lisa	\\ \hline
und	& und  & Andreas  & und	 & Gustav   & und    & Lisa	\\ \hline
	& 	   & 		  & drei & Andreas  & 	     & 	\\ \hline
FT 	&  FT  & Andreas  & FT 	 & Gustav   & Andreas& Andreas \\ \hline
zwei& drei & Andreas  & drei & Andreas  & zwei   & Lisa \\ \hline
drei& drei & Andreas  & drei & Gustav   & drei   & Lisa \\ \hline
\end{tabular}
\caption[Prädiktion mit Sprechererkennung]{Prädiktion mit Sprechererkennung}
\label{tab:predict_ALL_2}
\end{table}



\note{je größer die Pause zwischen den einzelnen Worten desto besser das Klassifikationsergebnis}