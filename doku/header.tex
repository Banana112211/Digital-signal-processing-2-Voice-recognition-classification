\documentclass[%
	12pt, 
	a4paper,
	oneside, 
	ngerman,
	bibtotoc,
	dvipsnames,
	table,
]{scrartcl}

\usepackage{wrapfig} 
\usepackage[printonlyused]{acronym}

%\usepackage{tikz}
\usepackage{amssymb,amsmath}
\usepackage{ifxetex,ifluatex}
\usepackage{fixltx2e} % provides \textsubscript
%\usepackage{struktex}

\ifxetex
  \usepackage{fontspec,xltxtra,xunicode}
  \defaultfontfeatures{Mapping=tex-text,Scale=MatchLowercase}
  \newcommand{\euro}{€}
    \usepackage{polyglossia} % -ak-
  \setmainlanguage[%
  	spelling=new,%old
  	latesthyphen=true,%false
	%babelshorthands=true,%false
	]{german} % -ak-
  %\setmainlanguage[variant=american]{english} % -ak-
  \usepackage{xecolor}
\else
  \ifluatex
    \usepackage{fontspec}
    \defaultfontfeatures{Mapping=tex-text,Scale=MatchLowercase}
    \newcommand{\euro}{€}
    \usepackage[ngerman]{babel} % -ak-
    \usepackage{color}
  \else
    \usepackage[utf8]{inputenc}
    \usepackage[TS1,T1]{fontenc} % -ak- T1 für \textdblquotedown
    \usepackage{lmodern} % -ak-
%   \usepackage{tgbonum}	% Bookman
%   \usepackage{tgpagella}	% Palatino
%   \usepackage{tgtermes}	% Times
%   \usepackage{tgschola}	% Century Schoolbook
%   \usepackage{iwona}	% 
%   \usepackage{anttor}
    \usepackage[ngerman]{babel}
    \usepackage{color}    
  \fi
\fi


% Format A4
\typearea[3mm]{13}

% Format A5
%\setlength{\paperwidth}{14.8cm}
%\setlength{\paperheight}{21cm}
%\typearea[2mm]{13}

% Format Taschenbuch
%\setlength{\paperwidth}{12cm}
%\setlength{\paperheight}{19cm}
%\typearea[2mm]{12}

% Format iPad
%\setlength{\paperwidth}{14,5cm}
%\setlength{\paperheight}{19cm}
%\typearea{13}

% Format iPod
%\setlength{\paperwidth}{9cm}
%\setlength{\paperheight}{11,5cm}
%\typearea{14}



\ifxetex
\else
	\usepackage{microtype}
\fi
\raggedbottom
\emergencystretch 0.8em

\definecolor{gcolor}{rgb}{0,0,0} % schwarz

%\usepackage{natbib}
%\bibliographystyle{plainnat}

%\usepackage{biblatex}

%\bibliography{$biblio-files$}

%\usepackage{fancyvrb}

% Redefine labelwidth for lists; otherwise, the enumerate package will cause
% markers to extend beyond the left margin.
\makeatletter\AtBeginDocument{%
  \renewcommand{\@listi}
    {\setlength{\labelwidth}{4em}}
}\makeatother
\usepackage{enumerate}

\usepackage{ctable}
\usepackage{float} % provides the H option for float placement

\usepackage{url}

\usepackage{graphicx}



% We will generate all images so they have a width \maxwidth. This means
% that they will get their normal width if they fit onto the page, but
% are scaled down if they would overflow the margins.
%\makeatletter
%\def\maxwidth{\ifdim\Gin@nat@width>\linewidth\linewidth
%\else\Gin@nat@width\fi}
%\makeatother
%\let\Oldincludegraphics\includegraphics
%\renewcommand{\includegraphics}[1]{\Oldincludegraphics[width=\maxwidth]{#1}}

\usepackage{verbatim}

\usepackage{nameref}


\newcommand{\seclabel}[1]{
\label{sec:#1}
}

\newcommand{\secnameref}[1]{
\nameref{sec:#1}
}

\newcommand{\secnamerefn}[1]{
\ref{sec:#1}: \nameref{sec:#1}
}

\newcommand{\secnamerefnpb}[1]{
Kapitel\secnamerefn{#1}auf Seite \pageref{sec:#1}
}

\newcommand{\secnamerefnpbf}[1]{
Kapitel\secnamerefn{#1}auf Seite \pageref{sec:#1}.
}

\newcommand{\secnamerefnp}[1]{
(Kapitel\secnamerefn{#1}auf Seite \pageref{sec:#1})
}



\usepackage{float}
\floatstyle{boxed}
\restylefloat{figure}

\newcommand{\figref}[1]{
Abb.~\ref{fig:#1}
}

\newcommand{\figreflong}[1]{
Abbildung~\ref{fig:#1}
}

\newcommand{\figrefp}[1]{
\figref{#1} auf Seite~\pageref{fig:#1}
}


% @acuda
% change to alternative figure numbering
% default:  in book class figures are numbered per chapter
%           in article class figures are numbred continuously
%
% change book-class default to continuosly behavior:
% \usepackage{chngcntr}
% \counterwithout{figure}{chapter}
%
% change article-class default to per chapter (section) style:
\usepackage{chngcntr}
%\counterwithin{figure}{section}


% @acuda
% generate new appendix behavior
% now add caption to toc without numbering
\let\Oldappendix\appendix
\def\appendix{
	\Oldappendix
	%\phantomsection 
	\addcontentsline{toc}{section}{Anhang}
	\renewcommand\refname{Anhang} \section*{Anhang}
}

% @acuda #####################
%add for centered captions following options: justification=justified, singlelinecheck=false
\usepackage[font=small, labelfont=bf]{caption}
\usepackage{subcaption}

\KOMAoptions{parskip=half}
%\addtokomafont{caption}{\footnotesize} %lable captions smaller...


\usepackage{blindtext} 

\usepackage{bibgerm}
%\usepackage{titlesec}	%clearpage (newpage with correct floating enviroment) before sections
\newcommand{\sectionbreak}{\clearpage}



\ifxetex
  \usepackage[setpagesize=false, % page size defined by xetex
              unicode=false, % unicode breaks when used with xetex
              xetex,
              bookmarks=true,
              pdfauthor={$author-meta$},
              pdftitle={$title-meta$},
              colorlinks=true,
              urlcolor=blue,
              linkcolor=blue]{hyperref}
\else
  \usepackage[unicode=true,
              bookmarks=true,
              pdfauthor={$author-meta$},
              pdftitle={$title-meta$},
              colorlinks=false,
              urlcolor=blue,
              linkcolor=blue]{hyperref}
\fi
\hypersetup{breaklinks=true, pdfborder={0 0 0}}

\setlength{\emergencystretch}{3em}  % prevent overfull lines


\usepackage{textcomp}

\usepackage{csquotes}

\usepackage{multirow}
\usepackage{pdflscape}

\setcounter{tocdepth}{3}


%%%% Eigene Kopf und Fußzeile %%%%%%%%%%%%%%%% 
%\usepackage{scrpage2}%                              Package laden 
%\pagestyle{scrheadings}%                           SCRPAGE2 Style aktivieren 
%\clearscrheadfoot%                                    lösche alle Kopf und Fußzeilen 
%\setheadwidth{textwithmarginpar}%             für textbreite+rand 
%\automark{chapter}%                                    Autoerkenneung Chapter 
%\ohead{\textbf{\pagemark}}%                      Seitenzahl 
%\renewcommand{\chaptermark}[1]{\markright{\ #1}} %löscht die Nummerierung von Chapter 
%\ihead{\textbf{\rightmark}} 
%%%%%%%%%% Striche Kopfzeile %%%%%%%%%%%%% 
%\setheadtopline{2pt}[\color{blue}] 
%\setheadsepline{1pt}[\color{blue}] 



\usepackage[%
	automark,
	headsepline,                %% Separation line below the header
	%footsepline,               %% Separation line above the footer
	markuppercase
]{scrpage2}
\automark[%
	%subsection  % durch renewcommand gelöst?
]{section}
\pagestyle{scrheadings}
\renewcommand{\sectionmark}[1]{\markright{\ #1}} %löscht die Nummerierung von Chapter 

\lefoot{}                      %% Bottom left on even pages
\lofoot{}                      %% Bottom left on odd pages
\refoot{}                      %% Bottom right on even pages
\rofoot{}                      %% Bottom right on odd pages
\cfoot{--~\pagemark~--}        %% Bottom center
 
%\lehead{\bfseries\pagemark}    %% Top left on even pages
%\lohead{\bfseries\headmark}    %% Top left on odd pages
%\rehead{\bfseries\headmark}    %% Top right on even pages
\rohead{\headmark}    %% Top right on odd pages
\chead{}                       %% Top center
%\renewcommand{\subsectionmark}[1]{\markright{\ #1}} %löscht die Nummerierung von Chapter 


%%%%%%%%%%%%%%%%%%%%%%%%%%%%%%%%%%%%%%%%%%%%%%%%%%%%%
% C M D :   E I G E N N A M E N
%%%%%%%%%%%%%%%%%%%%%%%%%%%%%%%%%%%%%%%%%%%%%%%%%%%%%

\newcommand{\en}[1]{%
\glqq\textit{#1}\grqq%
}


\usepackage{listings}
%\input{listings-python-setup}
%\input{python-function-definition-setup}

\usepackage{setspace}	%zeilenabstand

\makeatletter
\newcommand{\MSonehalfspacing}{%
  \setstretch{1.44}%  default
  \ifcase \@ptsize \relax % 10pt
    \setstretch {1.448}%
  \or % 11pt
    \setstretch {1.399}%
  \or % 12pt
    \setstretch {1.433}%
  \fi
}

\newcommand{\MSdoublespacing}{%
  \setstretch {1.92}%  default
  \ifcase \@ptsize \relax % 10pt
    \setstretch {1.936}%
  \or % 11pt
    \setstretch {1.866}%
  \or % 12pt
    \setstretch {1.902}%
  \fi
}
\makeatother


\onehalfspacing



%%%%%%%%%%%%%%%%%%%%%%%%%
% TEXT ANNOTATIONS		
%%%%%%%%%%%%%%%%%%%%%%%%%

\newcommand{\note}[1]{
\textcolor{orange}{\textbf{Notiz:} #1}
}

\makeatletter
\newcommand{\todo}[1]{
\textcolor{blue}{\textbf{ToDo:} #1}
}




\let\oldcite\cite%
\renewcommand{\cite}[2][]{%
	\ifx&#1&%
		\mbox{\oldcite{#2}}%
	\else%
		\mbox{\oldcite[#1]{#2}}%
	\fi%
}

\newcommand*{\eqcite}[2][]{%
	\vadjust{%
	    \smallskip
	    \hbox to \linewidth{\hfill%
			\ifx&#1&%
				\cite{#2}%
			\else%
				\cite[#1]{#2}%
			\fi%
	    }%
	}%
}%